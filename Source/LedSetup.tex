Although relatively radiation tolerant, lead tungstate scintillating crystals do lower the light output when exposed to radiation and recover when the radiation source is removed, trough spontaneus thermal-annealing mechanisms. Extensive studies performed at the Institute for High Energy Physics (IHEP) in Protvino, Russia, confirmed that the PbWO$_4$ light output changes with the irradiation dose rate. In particular, dedicated measurements showed that degradation of light output in PbWO$_4$ crystals occurs due to light transmission loss only, rather than changes in the intrinsic scintillation mechanism \cite{1}. Further complications arise because at the same irradiation intensity, changes in light output may vary from one crystal to another \cite{2} \cite{3}. \textcolor{red}{Add plot of radiation damage measurement here - and quote S.Fegan paper? Probably yes, since it also shows the effect of light annealing}

In order to preserve the intrinsic ECal energy resolution, the response of the crystals has to  be continuously monitored and, if necessary, recalibrated. To this end, a custom, LED-based monitoring system was specifically designed and installed in the detector setup after the 2012 test run.


The LED monitoring system setup is as follows:

In the future studies
will be carried to study how they can help crystals to recover from radiation
damage.


Since crystals are coupled to a light sensor, an APD in our case, the monitoring system should be able to test the response of the whole chain: crystal, APD, redout electronics
rather than to only measure the change in light transmission. In particular it will be used for monitoring the channel matching and the crystal transparency, the optical contact between crystals and sensors, preamplifier gain and stability, the linearity of the entire electronic section and the timing.
Since the PbWO4 light yield and the APD gain depend on the temperature a stable formalization ,with $\Delta T=0.1^\circ C$, is required.

%\bibitem{1} V.A. Batarin, et al., Nucl. Instr. and Meth. A 540 (2005) 131 (e-Print
%ArXiv physics/0410133).
%\bibitem{2} V.A. Batarin, et al., Nucl. Instr. and Meth. A 512 (2003) 484 (e-Print
%ArXiv hep-ex/0210011).
%\bibitem{3} V.A. Batarin, et al., Nucl. Instr. and Meth. A 550 (2005) 543 (e-Print
%ArXiv physics/0504085).
